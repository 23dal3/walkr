\documentclass[a4paper]{report}
\usepackage[utf8]{inputenc}
\usepackage[T1]{fontenc}
\usepackage{RJournal}
\usepackage{amsmath,amssymb,array}
\usepackage{booktabs}
\usepackage{float}
\usepackage{Sweave}
\usepackage[parfill]{parskip}

%\VignetteEngine{Sweave}
%\VignetteIndexEntry{walkr}

\begin{document}

%% do not edit, for illustration only
\sectionhead{Contributed research article}
\volume{XX}
\volnumber{YY}
\year{20ZZ}
\month{AAAA}

\begin{article}

\title{walkr}

\author{by David Kane, Andy Yao}

\maketitle      

\abstract{
The \texttt{walkr} package samples points using random walks from the intersection of 
the $N$ simplex with $M$ hyperplanes. Mathematically, the sampling space is all vectors $x$ 
that satisfy $Ax=b$, $\sum x = 1$, and $x_i \geq 0$. The sampling algorithms implemented 
are hit-and-run and Dikin walk, both of which are MCMC (Monte-Carlo Markov Chain) random 
walks. \texttt{walkr} also provide tools to examine and visualize the convergence
properties of the random walks.

}

\section{Introduction} 

\section{Sampling space: simple 3D case}

\section{Random Walks}

\subsection{Starting Points}

\subsection{Hit-and-run}

\subsection{Dikin Walk}

\section{Using \texttt{walkr}}

\section{Examining/Visualizing Results}

\section{Conclusion}

\section{Authors}

\address{David Kane\\
Managing Director \\
Hutchin Hill Capital\\
101 Federal Street, Boston, USA\\}
\email{dave.kane@gmail.com}

\address{Andy Yao\\
Mathematics and Physics\\
Williams College \\
Williamstown, MA, USA\\}
\email{ay3@williams.edu}

\end{article}
\end{document}
